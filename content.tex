\section{Executive summary}

\id{Summarize the assignment, how it went, general impressions, etc. Delete these blue comments from the final version of your report.}

\section{Requirements traceability}

\begin{table}[h]
    {\small
    \begin{tabular}{@{}p{2cm}p{2cm}p{2cm}p{8cm}@{}}
        \toprule
        \multicolumn{1}{c}{\textbf{Req ID}} & \multicolumn{1}{c}{\textbf{Status}} & \multicolumn{1}{c}{\textbf{Implemented in}} & \multicolumn{1}{c}{\textbf{Design considerations}} \\
        R1.1 & \{Missing, Partial, Implemented\} & \code{MapSetup.java} & The software sets up the same map by a hard-wired specification. The specification is given in the \code{MapSetup} class' \code{tiles} data structure and used in the \code{generateMap()} method at the initialization phase of the program. \\
        ... & ... & ...  & ... \\ \bottomrule
    \end{tabular}}
\end{table}

\id{The goal of this matrix is to aid your TA verify the satisfaction of requirements. If the table is superficial or too verbose, your TA will have a hard time verifying the completeness of your deliverable, and that will result in lost marks.
Use the Mising-Partial-Implemented status markers in your final report. Use the code{} command to format file names and Java code elements, such as class names, method names, etc.}

\section{Design}

\id{Elaborate on the design phase. Present models if applicable.}

\section{Implementation}

\id{Elaborate on the implementation phase.}

\section{Testing and QA}

\id{Elaborate on how you validated the deliverable.}

\section{Reflections}

\id{Answer the questions in the assignment.}

\section{Roles and responsibilities}

The team members contributed equally to the deliverable. \id{This is the ideal situation, but in case the workload has not been equal, please, report accordingly.}

\begin{itemize}
    \item \TODO{\textbf{John Doe} contributed to the conceptual design and the implementation of RQ1.1.}
    \item \TODO{...}
\end{itemize}